\section{Localization}
\label{sec:loc}
\textit{Finally, a method to compute the state (configuration) of the robot is required. The method should be applied to a real system such as the Nexus platform from the course and due to the size of the university it is important that the robot is able to use features to precisely measure its whereabouts. These features could be based on the Hokoyo 2D laser scanner mounted on the robot.\\
Again document what algorithm, and test how well it performs. You should at least write what model you choose for the robot and show that the localization works better than odometry alone.
Note: The map ”complete\_map\_project.pgm” use a scale 10pixel:1m.}

\subsection{Funktioner}
\todo[inline]{Noget kort omkring de forskellige funktioner vi har implementeret.}

\subsubsection{Funktioner}
\subsubsection{Funktioner}
\subsubsection{Funktioner}