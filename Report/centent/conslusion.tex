\section{Konklusion}
\label{sec:kon}
Implementeringen af \emph{Wave Front Planner} algoritmen fungere særdeles godt. Den er særlig god fordi den kan foretages fra flere forskellige startpositioner.\\

Planning algoritmens implementering virker ikke. Algoritmen vil kunne have optimeret robottens path og derved ville robotten køre signifikant færre kilometer. \\

Hvis rengøringen og kop opsamlingen skal ordnes af rengøringspersonale, vil samme antal af timer koste 483,68 €. Dette er beregnet ud fra en bevægelseshastighed matchende robotten. Dog vil et menneske være bedre end vores nuværende implementeringen til at løse TSM-problematikker. Dog er lønnen et kontinuert løbet som skal betales, hvor robottens pris er et engangsbeløb. \\


\section{Fremtidig arbejde}
\label{sec:frem}
Hvis resultatet skulle videreudvikles ville følgende fokuspunkter udarbejdes: 
\begin{itemize}
\item Udvide firkantopdelingen således der kan blive lavet regioner i ikke kvadratiske rum. Denne kunne gøres med en \emph{Brushfire} algoritme.
\item Få implementeret den graf baseret planner algoritme.
\item Skrive en driver til Hokoyo scanneren og hverved implementere Ransac algoritmen på NEXUS robotten.
\end{itemize}

