\section{Planning}
\label{sec:plan}
\textit{All rooms in the map must be checked in order to find cups. The robot must be within 2 meters of a cup in order to actually detect the cup and within 1 meter in order to collect it. Cups are marked in the map using one pixel with grayscale value 150. Cups can be offloaded at the two offloading stations in the cantina. The offloading stations are represented with pixel values 100. The robot must start and end at an offloading station.
You are free in regard in choice of algorithms. However, please document what algorithm you choose, how many kilometres the robot moves and how long it takes to calculate the path the robot takes.}

\subsection{Funktioner}
I

\subsubsection{Wave-Front Planner}
Som er angivet i problemstilling skal robotten starte og slutte ved en af de to aflæsningstationer ved kantinen efter den har været ude og samle alle kopperne. Derfor er det valgt at implementere en \emph{Wave-Front Planner} algoritme baseret på en \emph{std::queue} og \emph{four-point connectivity}. 
Dens egenskaber sikre at robotten kan finde tilbage til den tætteste aflæsningssted uanset om robotten befinder sig i den ene eller i den anden ende af TEK. \\

Før Wave-Front Planner aktiveres oprettes en to-dimensionelt table matchende dimensionerne af original billedet. Herefter overføres informationer om \emph{obstacles} og \emph{free space} til den nye tabel med pixel-værdier på hhv. 1 og 0. For undgåelse af kollidering af robot og \emph{obstacles}, forstøres \emph{obstacles} med radiusen af robotten, fire pixels.\\
Herefter pushes de to aflæsningspostioner, med pixel-værdien 100, ind i køen. \\

Hver position i køen bliver tjekket i forhold til \emph{four-point connectivity} og hvis positionen er inden for originalbilledet og har en pixel-værdi på 0, pushes den nye position til køen. Sådan bliver det generet en distance baseret sti til robotten.\\
Implementeringen kan ses i appendix \ref{app:wave}.
\todo[inline]{how many kilometres the robot moves and how long it takes to calculate the path the robot takes??}