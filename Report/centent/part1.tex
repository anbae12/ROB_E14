\section{Planning}
\label{sec:plan}
\textit{All rooms in the map must be checked in order to find cups. The robot must be within 2 meters of a cup in order to actually detect the cup and within 1 meter in order to collect it. Cups are marked in the map using one pixel with grayscale value 150. Cups can be offloaded at the two offloading stations in the cantina. The offloading stations are represented with pixel values 100. The robot must start and end at an offloading station.
You are free in regard in choice of algorithms. However, please document what algorithm you choose, how many kilometres the robot moves and how long it takes to calculate the path the robot takes.}

\subsection{Funktioner}
Plangægning af coverage af gulvarealet foregår i fire trin:
\begin{itemize}
	\item Generering af distancekort over arealet fra de to aflæsningspositioner.
	\item Opdelingen af arealet i regioner og returnering af diagonalkoordinater.
	\item \textcolor{red}{Opret graf indeholdene alle regioner.}
	\item \textcolor{red}{Planlæg rute igennem grafen således at robotten besøger alle regioner mindst én gang.}
	\item Få robotten til den næste region. Herefter vask og samle kopperne.
\end{itemize}

Punkterne markeret med rød var den oprindelige tanke til at navigere robotten rundt. Det lykkedes at få implementeringen til at virke på et mindre del af kortet, men grundet tid var det ikke muligt at få implementeringen til at virke 100\% på det komplette kort.

\subsubsection{Wave-Front Planner}
Som er angivet i problemstilling skal robotten starte og slutte ved en af de to aflæsningstationer ved kantinen efter den har været ude og samle alle kopperne. Derfor er det valgt at implementere en \emph{Wave-Front Planner} algoritme baseret på en \emph{std::queue} og \emph{four-point connectivity}. 
Dens egenskaber sikre at robotten kan finde tilbage til den tætteste aflæsningssted uanset om robotten befinder sig i den ene eller i den anden ende af kortet. \\

Før Wave-Front Planner aktiveres oprettes én to-dimensionelt tabel med matchende dimensionerne af original billedet. Herefter overføres informationer om \emph{obstacles} og \emph{free space} til den nye tabel med pixel-værdier på hhv. 1 og 0. For undgåelse af kollidering af robot og \emph{obstacles}, forstørres \emph{obstacles} med radiusen af robotten, dvs. fire pixels.\\
Herefter pushes de to aflæsningspostioner, med pixel-værdien 100, ind i køen. \\

Hver position i køen bliver tjekket i forhold til \emph{four-point connectivity} og hvis positionen har en pixel-værdi på 0, pushes den nye position til køen. Idet en pixel-værdien 1 indeholder en obstacle og ikke pushes ind i køen vil \emph{Wave-Front Planner} algoritmen vil ikke kunne komme uden for bygningen. Herved er det generet en distance til det nærmeste aflæsningssted.

\subsubsection{Firkantopdeling og output af diagonalsæt}
Efter \emph{Wave-Front Planner} algoritmen er blevet eksekveret opdeles bygningens rum i firkanter og returnerer den øvre ventre koordinat og den nedre højre koordinat i en struct \emph{coordinatesPair()}. 
Denne funktion er bestående af to trin, i det første trin opdeles bygningen i firkanter og det næste trin returnere diagonalsættet fra hver generet firkant.

Til dannelsen af firkanter løbes hver pixel igennem af et dobbelt \emph{For Loop}, hvor hver pixel får tjekket dens otte omkring liggende pixels, \emph{findPone()}. 
Tjekket af de otte omkring liggende pixels er lavet vha. simple \emph{IF Conditions}. Hvis der er match på én af betingelserne, i dette tilfælde, (P\(_{1-8}\) i figur \ref{fig:principotte}), pushes den pågældende koordinat puls scenariet, vha. en struct \emph{coordinates()}, på en køen \emph{queue}. Herefter sættes pixel-værdien til \emph{pointColor}.\\
Herefter udvides hvert koordinat som findes på \emph{queue} i fire retninger; nord, syd, øst og vest, indtil en pixel-værdi på \emph{obstacle} eller på \emph{pointColor}. For hver iteration opdateres hver pixel-værdi til \emph{rectangularColor}. Dette foretages for hvert koordinat på \emph{queue} og udføres i \emph{stageOne()}. \\


I andet step laves der igen et tjek på hver pixel, \emph{findPtwo()}. Idet billedet nu er opdelt i firekanter tjekkes der kun for fire scenarier, P\(_{1, 9, 10, 11}\), figur \ref{fig:principotte}. 
Ligeledes hvis der er match, pushes det pågældende koordinat puls scenariet i køen vha. en struct \emph{coordinates()}.\\

Herefter løbes køen igennem hvor der udføres fire forskellige handling for de respektive scenarier P\(_{1, 9, 10, 11}\), \emph{stageTwo()}. Herunder beskrives kort de følgende funktionaliteter for hvert scenarie:

\begin{itemize}

\item P\(_{1}\): y-koordinaten inkrementeres indtil næste koordinat med pixel-værdien \emph{rectangularColor} opfyldes. Herefter inkrementeres x-koordinaten ligeledes indtil den næste koordinat med pixel-værdien \emph{rectangularColor} mødes. Den nuværende position er svarende til koordinaten (x1,y1) nederst til højre i firkanten. Koordinatsættet pushes ind i køen vha. en struct \emph{coordinatesPair(x-1,y-1,x1+1,y1+1)}.
\item P\(_{9}\): x-koordinaten inkrementeres indtil næste koordinat med pixel-værdien \emph{rectangularColor} opfyldes, (x1,y1). 
Koordinatsættet pushes ind i køen vha. en struct \emph{coordinatesPair(x-1,y-1,x1+1,y1+1)}.
\item P\(_{10}\): y-koordinaten inkrementeres indtil næste koordinat med pixel-værdien \emph{rectangularColor} opfyldes (x1,y1). Koordinatsættet pushes ind i køen vha. en struct \emph{coordinatesPair(x-1,y-1,x1+1,y1+1)}.
\item P\(_{11}\): Ved dette scenarie pushes koordinatsættet (x,y) ind i køen vha. en struct \emph{coordinatesPair(x-1,y-1,x+1,y+1)}.
\end{itemize}
Implementeringen nævnte funktioner findes i \emph{findPoints.h} og \emph{findPoints.cpp}.





\subsubsection{ \textcolor{red}{Grafer}}
For at navigere robotten rundt i bygningen blev det valgt at associere hvert \emph{coordinatesPair()} med en vertex og konstruere en graf på baggrund af disse. Grafen oprettes i fire trin:
\begin{itemize}
	\item \emph{addVertex():} Først oprettes hver enkelt vertex og indsættes i en liste, \emph{vertices[n]} således at de kan tilgås ved en simpel indeksering.
	\item \emph{createEdges():} Som navnet antyder sørger denne funktion for at oprette edges imellem alle tilstødende vertices. Hvorvidt en vertex grænser op mod en anden afgøres ved at undersøge forholdene imellem deres koordinater. 
\begin{figure}[h!]
	\label{fig:coordRelation}
	%\includegraphics[scale=1]{coordRelation}
	\caption{Eksempel på afgørelse om naboskab for vertex $V_0$}
\end{figure}
På figur \ref{fig:coordRelation} ses de fire mulige situationer der kan opstå ved oprettelse af edges til naboliggende vertices. Det gælder at: \\
\scriptsize\emph{(Læs: $V_1$ er nabo til $V_0$ hvis og kun hvis koordinatet $(X_{01},Y_{00})$ eksisterer som top koordinat i den n'de vertex i listen vertices.)}
\normalsize	 	 
	 \begin{align*}
	 V_1\parallel V_0 & \equiv(X_{01},Y_{00})\in v[n].\text{top}(x,y)	 \\
	 V_2\parallel V_0 & \equiv(X_{01},Y_{00})\in v[n].\text{bottom}(x,y)\\
	 V_3\parallel V_0 & \equiv(X_{00},Y_{01})\in v[n].\text{bottom}(x,y)\\
	 V_4\parallel V_0 & \equiv(X_{00},Y_{01})\in v[n].\text{top}(x,y)\\
	 \end{align*}
Den faktiske implementering af dette kan ses i appendix \ref{app:createEdges}.
	 \item \emph{createPathGrid():} Denne funktion opretter en liste, \emph{pathGrid} over shortest-path til alle andre vertices i grafen. At opretholde alle disse \emph{pathGrid}'s er meget dyrt i memory, men desværre nødvendig for at denne implementering af en pathfinder opretter en path der overholder den 4-point-connectivity robotten kræver.
	 \item \emph{generatePath():} Slutteligt oprettes den path robotten skal følge for at besøge alle vertices. Til dette anvendes en modificeret shortest path first algoritme som set i appendix \ref{app:generatePath}. En vertex pushes på en kø og algoritmen initialiseres. Hver vertex pusher sine naboer på køen og bliver føjet til path. Det undersøges om den nuværende vertex og det forreste element i køen er naboer, hvis dette er tilfældet fortsætter algoritmen. Hvis ikke, findes ruten fra den nuværende vertex til det forreste element i køen ved hjælp af \emph{pathGrid} fundet i det sidste step. Når ruten er fundet føjes den til path. Når algoritmen har kørt vil der være oprettet en kø indeholdene indexet et på alle de regioner robotten skal besøge på sin rute.
\end{itemize}
Det er værd at nævne at problemet der skal løses her er et velkendt problem kaldet \textit{"Traveling Salesman Problem"}[TSP]. Formuleringen af dette problem er: Givet en række byer, hvad er den kortest mulige rute en sælger skal følge for at nå til alle byer. TSP er NP-Hard hvilket betyder at der ikke eksisterer en "pæn" løsning på problemet. \\~\\


\subsubsection{Flytning af robot til ny region}

Idet det ikke lykkedes at få en komplet graf implementering af alle regionerne, var næste løsning at implementere en \emph{Wave-Front Planner} algoritme som finder den korteste rute mellem nuværende position og start positionen på den næste region.




