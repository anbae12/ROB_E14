\appendices
\section{Planning}
\subsection{Wave-Front Planner}
\label{app:wave}

\todo[inline]{missing code}
\newpage
\subsection{Implementering af grafer}
\label{app:graf}
\subsection{pathPlanner.h}
\begin{lstlisting}
#ifndef PATHPLANNER_H
#define PATHPLANNER_H

#include <utility>
#include <vector>
#include <iostream>
#include <iterator>
#include <algorithm>
#include <queue>
#include <limits>
#include <list>

using namespace std;

class pathPlanner
{
	public:
   		pathPlanner();
    	~pathPlanner();
    	void addVertex(unsigned int topX, unsigned int topY, unsigned int bottomX, unsigned int bottomY);
    	void getPath();
	private:
   		const static unsigned int infinityCost = numeric_limits<unsigned int>::max();

    	struct coord {
        	unsigned int x = 0, y = 0;
    	};

		struct vertex {
        	list<unsigned int> adj;
        	unsigned int dist;
       		unsigned int pathGrid[24];
        	unsigned int identifier = 0;

        	coord top, bottom;
        	void setCoord(unsigned int topX, unsigned int topY, unsigned int bottomX, unsigned int bottomY) {
           		top.x = topX;
            	top.y = topY;
            	bottom.x = bottomX;
    	    	bottom.y = bottomY;
        	}

        	vertex() {
        		for (unsigned int i = 0; i < 24; i++)
		    		pathGrid[i] = infinityCost;
    		}
    	};

    	typedef vector<vertex> vertices_t;
    	vertices_t vertices;

    	list<unsigned int> path;
 
    	void createEdges();
    	void createPathGrids();
		void generatePath();
};

#endif // PATHPLANNER_H

\end{lstlisting}~\\
\subsection{pathPlanner.c : void getPath()}
\label{app:getPath}
\begin{minipage}{\linewidth}
\begin{lstlisting}
void pathPlanner::getPath()
{
    createEdges();
    createPathGrids();
    generatePath();

    while (!path.empty()) {
        cout << "x: " << vertices[path.front()].top.x << " y: " << vertices[path.front()].top.y << endl;
        path.pop_front();
    }
}
\end{lstlisting}
\end{minipage}
\subsection{pathPlanner.c : void createPathGrid()}
\label{app:createPathGrid}
\begin{minipage}{\linewidth}
\begin{lstlisting}
void pathPlanner::createPathGrids()
{
    for (vertices_t::iterator vertex_it = vertices.begin(); vertex_it != vertices.end();
    																			 vertex_it++) {
    	//for all vertices check distance to neighbor and update pathGrid if not previously visited
        for (list<unsigned int>::iterator adj1_it = vertices[vertex_it->identifier].adj.begin(); 
        						adj1_it != vertices[vertex_it->identifier].adj.end(); adj1_it++) {
            if (vertex_it->pathGrid[*adj1_it] == infinityCost)
                vertex_it->pathGrid[*adj1_it] = vertex_it->identifier;
            //check distance to neighbors of neighbors and update pathGrid if not previously visited
            for (list<unsigned int>::iterator adj2_it = vertices[*adj1_it].adj.begin(); 
            									adj2_it != vertices[*adj1_it].adj.end(); adj2_it++) {
                if (vertex_it->pathGrid[*adj2_it] == infinityCost)
                    vertex_it->pathGrid[*adj2_it] = *adj1_it;

				//check distance to neighbors of neighbors neighbors (wtf?) 
				//and update pathGrid if not previously visited
                for (list<unsigned int>::iterator adj3_it = vertices[*adj2_it].adj.begin(); 
                								adj3_it != vertices[*adj2_it].adj.end(); adj3_it++) {
                    if (vertex_it->pathGrid[*adj3_it] == infinityCost)
                        vertex_it->pathGrid[*adj3_it] = *adj2_it;
                }
            }
        }
    }
}
\end{lstlisting}~\\
\end{minipage}
\subsection{pathPlanner.c : void generatePath()}
\label{app:generatePath}
\begin{minipage}{\linewidth}
\begin{lstlisting}
void pathPlanner::generatePath()
{
    unsigned int currentVertex = 0, destinationVertex = 0, neighbor = 0;
    queue<unsigned int> q;
    list<unsigned int> queuePathGrid;

    //Set all costs to infinity and path to zero
    for (vertices_t::iterator vertex_it = vertices.begin(); vertex_it != vertices.end(); vertex_it++)
        vertex_it->dist = infinityCost;

    //Start at vertex 0
    vertices[currentVertex].dist = 0;
    q.push(currentVertex);

    while (!q.empty()) {
        currentVertex = q.front();
        q.pop();

        if (!q.empty()) {
            neighbor = 0;
            for (list<unsigned int>::iterator ajd_it = vertices[q.front()].adj.begin();
            									ajd_it != vertices[q.front()].adj.end(); ajd_it++)
                if (q.front() == *ajd_it)
                    neighbor = 1;

            if (!neighbor){
                queuePathGrid.clear();
                destinationVertex = q.front();
                while (destinationVertex != currentVertex) {
                    destinationVertex = vertices[currentVertex].pathGrid[destinationVertex];
                    queuePathGrid.push_back(destinationVertex);
                }
                queuePathGrid.reverse();
                while (!queuePathGrid.empty()){
                    path.push_back(queuePathGrid.front());
                    queuePathGrid.pop_front();
                }
            }
        } else
            path.push_back(currentVertex);


        for (list<unsigned int>::iterator ajd_it = vertices[currentVertex].adj.begin(); 
        										ajd_it != vertices[currentVertex].adj.end(); ajd_it++) {
            if (vertices[*ajd_it].dist == infinityCost)
            {
                vertices[*ajd_it].dist = vertices[currentVertex].dist + 1;
                q.push(*ajd_it);
            }
        }
    }
}
\end{lstlisting}~\\
\end{minipage}
\subsection{pathPlanner.c : void createEdges()}
\label{app:createEdges}
\begin{minipage}{\linewidth}
\begin{lstlisting}
void pathPlanner::createEdges()
{

    unsigned int vertexIndex = 0, neighborIndex = 0;

    //Run throug every vertex to find it's edges
    for (vertices_t::iterator vertex_it = vertices.begin(); vertex_it != vertices.end();
    																		 vertex_it++){
		// Get the vertex index/number
        vertexIndex = distance(vertices.begin(), vertex_it);
        vertex_it->identifier = vertexIndex;
 
        //Run throug every vertex again to see if it is a neighbor
        for (vertices_t::iterator neighbor_it = vertices.begin(); 
        								neighbor_it != vertices.end(); neighbor_it++) {
    		// Get the neighbor index/number
            neighborIndex = distance(vertices.begin(), neighbor_it); 
            
            //Only check if it is not itself
            if (neighborIndex != vertexIndex) {
                //Check right and add egde and cost
                if (vertex_it->bottom.x + 2 == neighbor_it->top.x && 
                					vertex_it->top.y == neighbor_it->top.y){
                    vertex_it->adj.push_back(neighborIndex);
				}
				
                //Check left and add egde and cost
                if (vertex_it->top.x - 2 == neighbor_it->bottom.x && 
                					vertex_it->bottom.y == neighbor_it->bottom.y){
                    vertex_it->adj.push_back(neighborIndex);
				}
				              
                //Check top and add egde and cost
                if (vertex_it->bottom.x == neighbor_it->bottom.x && 
                					vertex_it->top.y - 2 == neighbor_it->bottom.y){
                    vertex_it->adj.push_back(neighborIndex);
				}
				
                //Check bottom and add egde and cost
                if (vertex_it->top.x == neighbor_it->top.x && 
                					vertex_it->bottom.y + 2 == neighbor_it->top.y){
                    vertex_it->adj.push_back(neighborIndex);
                }
            }
        }
    }
}
\end{lstlisting}~\\
\end{minipage}
\subsection{pathPlanner.c : void addVertex()}
\label{app:addVertex}
\begin{lstlisting}
void pathPlanner::addVertex(unsigned int topX, unsigned int topY, 
										unsigned int bottomX, unsigned int bottomY)
{
    //Insert all vertices in a table so it is associated with a number/index
    vertex temp;
    temp.setCoord(topX, topY, bottomX, bottomY);
    vertices.push_back(temp);
}
\end{lstlisting}~\\

\newpage
\section{Coverage}
\subsection{Firkantopdeling og output af diagonalsæt}
\label{app:firkant}

\begin{figure}[!th]
\centering
\begin{tikzpicture}[scale=0.95]
\include*{./graphics/findPoints}
\end{tikzpicture}
\caption[tekst i indholdsfortegnelsen]{Princip tegning af tjek af otte omkring liggende pixels.}
\label{fig:principotte}
\end{figure}
\newpage
%\lstinputlisting{../Code/findDiagonalPair.cpp}
\todo[inline]{missing code}


\newpage
\section{Localization}
